\documentclass[13pt]{report}

%import des paquets
\usepackage{luatextra}
\usepackage[french]{babel}

\usepackage{fancyhdr}

\usepackage{lastpage}
\usepackage[a4paper, left=2cm, right=2cm, top=3cm, bottom=3cm]{geometry}

%semaine 2
\usepackage{hyperref} %liens internes
\usepackage{hhline} %meilleures lignes dans un tableau
\usepackage{multirow} %fusion de lignes dans un tableau
\usepackage{listings}

\usepackage{amsmath} %plus de maths

\pagestyle{fancy}

%infos sur le doc
\title{Mon premier rapport}
\author{Arthur \textsc{Blaise}}

%constantes
\makeatletter
\let\thetitle\@title
\let\theauthor\@author
\let\thedate\@date
\makeatother

%personnalisation en-têtes et footers
\fancyhead[C]{\thetitle}
\fancyhead[L]{\theauthor}
\fancyhead[R]{\thedate}

\fancyfoot[L]{\rightmark}
\fancyfoot[R]{\thepage /\pageref{LastPage}}
\fancyfoot[C]{}

\begin{document}
	\maketitle
	
	\tableofcontents
	
	\clearpage
	
	\section{Mes sites favoris}
	
	Voici la liste de mes sites favoris :
	\begin{itemize}
		\item Moteur de recherche : \href{http://google.fr}{Goooooogle}
		\item Site de vidéos : \href{https://youtube.com}{YouToube}
		\item Logiciel de discussion : \href{https://discord.gg}{Discoooord}
		\item Super lib : \href{https://frmc-lib.rtfd.io}{frmcLib}		
	\end{itemize}
	Et puis j'aime bien placer le symbole de \LaTeX \emph{un peu partout...}\\

	Et voilà une citation, pour faire joli : \cite{first}
	
	\clearpage
	
	\section{Mes résultats au MOOOOOC \LaTeX}
	
	\begin{table}[h]
		\begin{center}
			\begin{tabular}{|*{4}{c|}}
				\hline
				Intitulé & N° & Coef. & Note \\
				\hline
				 \multirow{4}{*}{QCM}& 01 &  1   & 1/1 \\
				 					 & 02 &  2   & 1/1 \\
				 					 & 03 &  2.5 & 2/1 \\
				 					 & 04 &  1   &  -  \\
				 \hline
				 TP					 & 01 &  4   &  -  \\
				 \hhline{|*{4}{=|}}
				\multicolumn{3}{|c|}{Moyenne}    & 1/1 \\
				 \hline
			\end{tabular}
			\caption{Résultats de moi-même}
		\end{center}
	\end{table}
	
	\clearpage
	
	\section{Une image ?}

	\begin{figure}[h]
		\begin{center}
			\includegraphics[scale=0.4]{../img/discord.png}
		\end{center}
		\caption{Logo de Discord}
	\end{figure}

	\clearpage
	
	\section{Let's make Python!}
	
	\lstinputlisting[language=Python, frame=double]{../code/w2.py}

	\clearpage
	
	\section{Mathématiques}
	
	Ici je met des mathématiques en ligne : \(\sqrt{42}\) en est un exemple.\\
	\\
	Et là, voici une équation toute seule : 
	
	\[
	\frac{\sqrt{2}+k_0}{e^{\pi}}
	\]
	\\
	On peut aussi créer un environnement, comme cela : 
	
	\begin{equation}
		e^{i\pi} + \alpha \cdot 2=0
	\end{equation}
	\\
	Et même des sommes ou des produits~!
	
	\begin{equation}
	\sum_{k=0}^{\infty}\frac{y^k}{\pi} = 7 \cdot k
	\end{equation}
	
	\begin{equation}
	n!=\prod_{k=1}^n k
	\end{equation}
	\\
	Ainsi que des intégrales....
	
	$$
	\int_a^b f(x) \, \mathrm{d}x
	$$
	\\
	Et aussi ça : \((A \leftrightarrow B) \leftrightarrow (A \leftarrow B ~ and ~ A \rightarrow B)\) \\
	\\
	Ou des matrices ? \emph{Why not!}

	$$	
	\left[
		\begin{array}{c c c}
		1 & 2 & 3 \\
		4 & 5 & 6 \\
		7 & 9 & 9
		\end{array}
	\right)
	$$ 
	\\
	Puis des accolades horizontales :
	
	\[
	\underbrace{x+ \cdots + x}_{42 ~\mathit{fois}}
	\]
	\\	
	Et des calculs alignés :
	\begin{align*}
		&& x^3 + y^2 &=1 \\
		&\leftrightarrow& y^2 &= 1-x^3 \\
		&\Leftrightarrow& y &= \sqrt{1-x^3}	
	\end{align*}
	
	
	\clearpage
	
	\section{Formule quadratique}
	
	\begin{align*}
	&&ax^2 + bx + c& = 0 \\
	&\rightarrow &x^2 + \frac{b}{a}x + \frac{c}{a}& = 0 \\
	&\leftrightarrow &x^2 + \frac{b}{a}x& = - \frac{c}{a}
	\end{align*}
	
	Application de la méthode de complétion du carré :
	
	\begin{align*}
	&& x^2 + \frac{b}{a}x + \left(\frac{b}{2a}\right)^2 &= - \frac{c}{a} + \left(\frac{b}{2a}\right)^2 \\
	&\leftrightarrow& \left(x + \frac{b}{2a}\right)^2   &= - \frac{c}{a} + \frac{b^2}{4a^2} \\
	&\Leftrightarrow& \left(x + \frac{b}{2a} \right)^2  &= \frac{b^2 - 4ac}{4a^2}
	\end{align*}
	
	Nous obtenons ainsi :
	
	\begin{align*}
	&&x + \frac{b}{2a}  &= \pm \frac{\sqrt{b^2 - 4ac}}{2a} \\
	&\leftrightarrow& x &= \frac{-b \pm \sqrt{b^2 - 4ac}}{2a}
	\end{align*}

	
	\bibliographystyle{plain}
	\bibliography{biblio}

\end{document}
